% Resume

\documentclass{article}

\usepackage{geometry}
\usepackage{multicol}
\usepackage{color}
\usepackage[none]{hyphenat} % To remove hyphenation from paragraph text
\usepackage{hyperref}
\usepackage{enumitem}

\geometry{top=1.5cm,bottom=1.5cm,left=1.5cm,right=1.5cm}
\hypersetup{colorlinks=true,linkcolor=blue,filecolor=magenta,urlcolor=blue}
% \urlstyle{same}
\pagestyle{empty}
\setlength\parindent{0pt}

\definecolor{darkcyan}{RGB}{0,139,139}
\definecolor{darkgray}{RGB}{100,100,100}
\definecolor{black}{RGB}{0,0,0}

\newenvironment{me}{\fontfamily{phv}\selectfont\bfseries\large}{\par}
\newenvironment{name}{\fontfamily{phv}\selectfont\bfseries\normalsize}{\par}
\newenvironment{colorheading}{\fontfamily{phv}\selectfont\bfseries\normalsize\color{darkcyan}}{\par}
\newenvironment{genericbody}{\fontfamily{phv}\selectfont\small\raggedright}{\par}
\newenvironment{metadata}{\fontfamily{phv}\selectfont\itshape\small\color{darkgray}}{\par}
\newenvironment{itemizedbody}{\fontfamily{phv}\selectfont\small\begin{itemize}}{\end{itemize}\par}

\newenvironment{info}[2]%
{%
  \begin{minipage}{0.20\linewidth}\begin{metadata}#1\end{metadata}\end{minipage}%
  \begin{minipage}{0.80\linewidth}\begin{genericbody}#2\end{genericbody}\end{minipage}%
}%
{\par}

\newenvironment{experience}[2]%
{%
  \begin{minipage}{0.6\linewidth}\begin{flushleft}#1\end{flushleft}\end{minipage}%
  \hfill%
  \begin{minipage}{0.3\linewidth}\begin{flushright}#2\end{flushright}\end{minipage}%
}%
{}

\newcommand{\lineitem}{\item[{$\to$}]}
\newcommand{\dotsep}{\begin{centering}\begin{colorheading}{$\bullet$}\end{colorheading}\end{centering}}

\begin{document}
  \begin{minipage}[t]{0.38\linewidth}
    \begin{me}
      Sekhar Bhattacharya
    \end{me}
    \begin{colorheading}
      Computer Engineer
    \end{colorheading}
    \bigskip
    \begin{info}{Mobile}{+1408 513 5898}\end{info}
    \begin{info}{Email}{sekhar.bx@pm.me}\end{info}
    \begin{info}{Github}{\url{http://github.com/0ctobyte}}\end{info}
    \smallskip
    \dotsep
    \smallskip
    \begin{experience}
    {
      \begin{name}Education\end{name}
      \begin{colorheading}University of Waterloo\end{colorheading}
    }
    {
      \begin{metadata}Graduated Summer 2016\end{metadata}
    }
    \end{experience}
    \begin{genericbody}
      Bachelor of Applied Science in Computer Engineering
    \end{genericbody}
  \end{minipage}
  \hfill
  \begin{minipage}[t]{0.6\linewidth}
    \begin{experience}
    {
      \begin{name}Apple Inc\end{name}
      \begin{colorheading}Silicon Validation\end{colorheading}
    }
    {
      \begin{metadata}Sept 2014 - Dec 2014\end{metadata}
      \begin{metadata}Sept 2014 - Dec 2015\end{metadata}
      \begin{metadata}Aug 2016 - Present\end{metadata}
    }
    \end{experience}
    \begin{itemizedbody}
      \lineitem Helped maintain and improve software tools used for the validation of Apple SoCs
      \lineitem Bring up said tools on silicon and FPGA and drive test plans to completion before tape-out deadlines
      \lineitem Developed test content (C/ARM/Forth) running directly on silicon/FPGA in order to stress CPU,
                memory hierarchy, power management features and various other SoC logic across all major Apple SoCs
      \lineitem Supported debugging of any and all failures encountered by validation tools (software and hardware)
      \lineitem Worked with RTL designers to root cause hardware failures in the SoCs as well as developing targeted
                test content to reproduce potential failures on silicon that were identified by design/formal
                verification teams
    \end{itemizedbody}
    \smallskip
    \begin{experience}
    {
      \begin{name}Evertz Microsystems Ltd\end{name}
      \begin{colorheading}Hardware Engineering Intern\end{colorheading}
    }
    {
      \begin{metadata}Jan 2014 - April 2014\end{metadata}
    }
    \end{experience}
    \begin{itemizedbody}
      \lineitem Implemented SPI and I\textsuperscript{2}C controllers in VHDL for use on Xilinx FPGAs
      \lineitem Used Chipscope to verify and test hardware synthesized from VHDL code on working products
      \lineitem Used tools such as Phabrix, oscilloscopes and waveform monitors to debug hardware issues
    \end{itemizedbody}
    \smallskip
    \begin{experience}
    {
      \begin{name}Blackberry Ltd\end{name}
      \begin{colorheading}Platform Software Developer Co-op\end{colorheading}
    }
    {
      \begin{metadata}May 2013 - Aug 2013\end{metadata}
    }
    \end{experience}
    \begin{itemizedbody}
      \lineitem Developed C code to test and bring-up hardware in prototype Blackberry devices
      \lineitem Helped implement a driver to communicate with a digital mic
      \lineitem Wrote test code to verify hardware interface buses such as UART and I\textsuperscript{2}C
      \lineitem Worked on a Blackberry app using C++ and Qt/QML
    \end{itemizedbody}
    %\smallskip
    %\begin{experience}
    %{
    %  \begin{name}Imlocal Inc\end{name}
    %  \begin{colorheading}Web Developer\end{colorheading}
    %}
    %{
    %  \begin{metadata}Sept 2012 - Dec 2012\end{metadata}
    %}
    %\end{experience}
    %\begin{itemizedbody}
    %  \lineitem Implemented a REST based API to connect app and back-end (PostgreSQL/Ruby on Rails) using JSON for
    %           data exchange
    %\end{itemizedbody}
    %\smallskip
    %\begin{experience}
    %{
    %  \begin{name}Canadian Bank Note Company Ltd\end{name}
    %  \begin{colorheading}Software Developer\end{colorheading}
    %}
    %{
    %  \begin{metadata}Jan 2011 - April 2011\end{metadata}
    %}
    %\end{experience}
    %\begin{itemizedbody}
    % \lineitem Worked on a web based security monitor using the Django framework (Python)
    %\end{itemizedbody}
    \smallskip
    \dotsep
    \smallskip
    \begin{name}Personal Projects\end{name}
    \smallskip
    \begin{experience}
    {
      \begin{colorheading}goodboy\end{colorheading}
      \begin{genericbody}\url{http://github.com/0ctobyte/goodboy}\end{genericbody}
    }
    {
      \begin{metadata}C++\end{metadata}
    }
    \end{experience}
    \begin{genericbody}
      Built an emulator for the original Nintendo GameBoy system. Implemented a CPU instruction emulator for the Sharp
      LR35902 microprocessor (z80 hybrid). Implemented accurate emulation of various other hardware devices (timers,
      LCD, PPU, DMA etc.).
    \end{genericbody}
    \medskip
    \begin{experience}
    {
      \begin{colorheading}procyon\end{colorheading}
      \begin{genericbody}\url{http://github.com/0ctobyte/procyon}\end{genericbody}
    }
    {
      \begin{metadata}SystemVerilog\end{metadata}
    }
    \end{experience}
    \begin{genericbody}
      Built a dynamic scheduling, scalar, speculative processor that runs on FPGA (Altera Cylone-IVe). Developed fully
      synthesizable hardware structures including reorder buffer, reservation stations, register renaming etc.
      Implemented load and store queues to handle out-of-order memory accesses + memory disambiguation. Built data
      cache and miss queue for hit-under-miss optimization.
    \end{genericbody}
    \medskip
    \begin{experience}
    {
      \begin{colorheading}raytracer\end{colorheading}
      \begin{genericbody}\url{http://github.com/0ctobyte/raytracer}\end{genericbody}
    }
    {
      \begin{metadata}C++\end{metadata}
      \begin{metadata}Lua\end{metadata}
    }
    \end{experience}
    \begin{genericbody}
      Generates a PNG of a 3D rendered scene described by a Lua based DSL. Implemented features such as bump/texture
      mapping, reflection, refraction, soft shadows and other algorithms. Developed algorithms to perform lighting
      computations and implemented multi-threading using pthreads. Gained a thorough understanding of 3D matrix/vector
      mathematics and raytracing algorithms.
    \end{genericbody}
    %\begin{section}
    %{
    %  \begin{colorheading}
    %    {$\bullet$}
    %  \end{colorheading}
    %}
    %\end{section}
    %\vspace*{0.5cm}
    %\begin{name}
    %  Areas of Interest
    %\end{name}
    %\begin{minipage}[t]{\columnwidth}
    %  \begin{metadata}
    %    Operating systems \newline
    %    CPU architecture \newline
    %    RTL Design and Validation \newline
    %    Embedded systems \newline
    %    3D graphics programming \newline
    %    C/C++/ARM assembly programming
    %  \end{metadata}
    %\end{minipage}
  \end{minipage}
\end{document}
