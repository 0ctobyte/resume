% Resume

\documentclass{article}

\usepackage{geometry}
\usepackage{multicol}
\usepackage{color}
\usepackage[none]{hyphenat} % To remove hyphenation from paragraph text
\usepackage{hyperref}

\geometry{top=2cm,bottom=2cm,left=1.5cm,right=1.5cm}
\hypersetup{colorlinks=true,linkcolor=blue,filecolor=magenta,urlcolor=blue}
% \urlstyle{same}
\pagestyle{empty}
\setlength\parindent{0pt}

\definecolor{darkcyan}{RGB}{0,139,139}
\definecolor{darkgray}{RGB}{100,100,100}
\definecolor{black}{RGB}{0,0,0}

\renewenvironment{section}[2][\hfill]{\begin{minipage}[t]{0.20\columnwidth}#1\end{minipage}\begin{minipage}[t]{0.80\columnwidth}#2\end{minipage}}{\par}
\newenvironment{heading1}{\fontfamily{phv}\selectfont\bfseries\large}{\par}
\newenvironment{heading2}{\fontfamily{phv}\selectfont\bfseries\normalsize}{\par}
\newenvironment{heading3}{\fontfamily{phv}\selectfont\bfseries\normalsize}{\par}
\newenvironment{body1}{\fontfamily{phv}\selectfont\small}{\par}
\newenvironment{body2}{\fontfamily{phv}\selectfont\itshape\small\begin{flushleft}}{\end{flushleft}\par}


\begin{document}

  \begin{multicols}{2}
  
    \begin{section}
    {
      \begin{heading1}
        Sekhar Bhattacharya
      \end{heading1}
    }
    \end{section}

    \begin{section}
    {
      \begin{heading3}
        \textcolor{darkcyan}{4B Computer Engineering}
      \end{heading3}
    }
    \end{section}

    \vspace*{0.5cm}

    \begin{section}
    [
      \begin{body2}
        \textcolor{darkgray}{Mobile}
      \end{body2}
    ]
    {
      \begin{body1}
        +1408 513 5898
      \end{body1}
    }
    \end{section}
    
    \begin{section}
    [
      \begin{body2}
        \textcolor{darkgray}{Email}
      \end{body2}
    ]
    {
      \begin{body1}
        s4bhatta@uwaterloo.ca
      \end{body1}
    }
    \end{section}

    \begin{section}
    [
      \begin{body2}
        \textcolor{darkgray}{Student \#}
      \end{body2}
    ]
    {
      \begin{body1}
        20381175
      \end{body1}
    }
    \end{section}

    \begin{section}
    [
      \begin{body2}
        \textcolor{darkgray}{Github}
      \end{body2}
    ]
    {
      \begin{body1}
        \url{http://github.com/0ctobyte}
      \end{body1}
    }
    \end{section}

    \vspace*{0.5cm}

    \begin{section}
    {
      \begin{heading3}
        \textcolor{darkcyan}{$\bullet$}
      \end{heading3}
    }
    \end{section}

    \vspace*{0.5cm}

    \begin{section}
    {
      \begin{heading2}
        Apple Inc
      \end{heading2}
    }
    \end{section}

    \begin{section}
    {
      \begin{heading3}
        \textcolor{darkcyan}{Silicon Validation}
      \end{heading3}
    }
    \end{section}

    \begin{section}
    [
      \begin{body2}
        \textcolor{darkgray}{Sept 2014 - Dec 2014}\newline
        \textcolor{darkgray}{$\bullet$}\newline
        \textcolor{darkgray}{Sept 2015 - Dec 2015}
      \end{body2}
    ]
    {
      \begin{body1}
        Worked with co-workers to maintain and improve existing software tools (C/ARM assembly/Forth) used for the validation of SoCs (System-on-a-Chip). 
        Worked with RTL designers and validation team members to root cause failures in the SoCs. Developed code that runs directly on silicon to test SoC logic.
      \end{body1}
    }
    \end{section}

    \vspace*{0.5cm}

    \begin{section}
    {
      \begin{heading2}
        Evertz Microsystems Ltd
      \end{heading2}
    }
    \end{section}

    \begin{section}
    {
      \begin{heading3}
        \textcolor{darkcyan}{Hardware Engineering Intern}
      \end{heading3}
    }
    \end{section}

    \begin{section}
    [
      \begin{body2}
        \textcolor{darkgray}{Jan 2014 - April 2014}
      \end{body2}
    ]
    {
      \begin{body1}
        Developed VHDL code for use on Xilinx FPGAs. Implemented SPI and I\textsuperscript{2}C controllers in VHDL.
        Used Chipscope to verify and test hardware synthesized from VHDL code on working products.
        Used tools such as Phabrix, oscilloscopes and waveform monitors to debug hardware issues.
      \end{body1}
    }
    \end{section}

    \vspace*{0.5cm}

    \begin{section}
    {
      \begin{heading2}
        Blackberry Ltd
      \end{heading2}
    }
    \end{section}

    \begin{section}
    {
      \begin{heading3}
        \textcolor{darkcyan}{Platform Software Developer Co-op}
      \end{heading3}
    }
    \end{section}

    \begin{section}
    [
      \begin{body2}
        \textcolor{darkgray}{May 2013 - Aug 2013}
      \end{body2}
    ]
    {
      \begin{body1}
        Developed C code to test and bring-up hardware in prototype Blackberry devices.
        Helped implement a driver to communicate with a digital mic.
        Wrote test code to verify hardware interface buses such as UART and I\textsuperscript{2}C.
        Worked on a Blackberry app using C++ and Qt/QML.
      \end{body1}
    }
    \end{section}

    \vspace*{0.5cm}

    \begin{section}
    {
      \begin{heading2}
        Imlocal Inc
      \end{heading2}
    }
    \end{section}

    \begin{section}
    {
      \begin{heading3}
        \textcolor{darkcyan}{Web Developer}
      \end{heading3}
    }
    \end{section}

    \begin{section}
    [
      \begin{body2}
        \textcolor{darkgray}{Sept 2012 - Dec 2012}
      \end{body2}
    ]
    {
      \begin{body1}
        Implemented a REST based API to connect app and back-end (PostgreSQL/Ruby on Rails) using JSON for data exchange.
      \end{body1}
    }
    \end{section}

    \vspace*{0.5cm}

    %\begin{section}
    %{
    %  \begin{heading2}
    %    Canadian Bank Note Company Ltd
    %  \end{heading2}
    %}
    %\end{section}

    %\begin{section}
    %{
    %  \begin{heading3}
    %    \textcolor{darkcyan}{Software Developer}
    %  \end{heading3}
    %}
    %\end{section}

    %\begin{section}
    %[
    %  \begin{body2}
    %    \textcolor{darkgray}{Jan 2011 - April 2011}
    %  \end{body2}
    %]
    %{
    %  \begin{body1}
    %    Worked on a web based security monitor using the Django framework (Python).
    %  \end{body1}
    %}
    %\end{section}

    %\vspace*{0.5cm}

    \begin{section}
    {
      \begin{heading3}
        \textcolor{darkcyan}{$\bullet$}
      \end{heading3}
    }
    \end{section}

    \vspace*{0.5cm}

    \begin{section}
    [
      \begin{body2}
        \textcolor{darkgray}{Expected Summer 2016}
      \end{body2}
    ]
    {
      \begin{heading2}
        University of Waterloo
      \end{heading2}
      \par
      \begin{heading3}
        \textcolor{darkcyan}{Bachelor of Applied Science in Computer Engineering}
      \end{heading3}
    }
    \end{section}

    \vfill
    \columnbreak
    \vspace*{4.15cm}

    \begin{minipage}[t]{\columnwidth}
      
      \begin{heading2}
        Personal Projects
      \end{heading2}

      \begin{heading3}
        \textcolor{darkcyan}{arm-kernel}
      \end{heading3}

      \begin{body1}
        \url{http://github.com/0ctobyte/arm-kernel}
      \end{body1}

      \begin{section}
      [
        \begin{body2}
          \textcolor{darkgray}{C \newline ARM}
        \end{body2}
      ]
      {
        \begin{body1}
          An attempt at writing a very simple 32-bit kernel for ARM processors using concepts learned from the ECE 254 (OS) course.
          Written in C and ARM assembly. Implemented page allocation, heap memory management, page table manipulation. Able to boot on real hardware (Beaglebone Black)! Work in progress.
        \end{body1}
      }
      \end{section}

      \vspace*{0.5cm}

      \begin{heading3}
        \textcolor{darkcyan}{raytracer}
      \end{heading3}

      \begin{body1}
        \url{http://github.com/0ctobyte/raytracer}
      \end{body1}

      \begin{section}
      [
        \begin{body2}
          \textcolor{darkgray}{C++ \newline Lua}
        \end{body2}
      ]
      {
        \begin{body1}
          Generates a PNG of a 3D rendered scene described by a Lua based DSL. Implemented features such as bump/texture mapping, reflection, refraction, soft shadows and other algorithms. Developed algorithms to perform lighting computations and implemented multi-threading using pthreads. Gained a thorough understanding of 3D matrix/vector mathematics and raytracing algorithms.
        \end{body1}
      }
      \end{section}

      \vspace*{0.5cm}
      
      \begin{heading3}
        \textcolor{darkcyan}{ece250ds}
      \end{heading3}

      \begin{body1}
        \url{http://github.com/0ctobyte/ece250ds}
      \end{body1}

      \begin{section}
      [
        \begin{body2}
          \textcolor{darkgray}{Python}
        \end{body2}
      ]
      {
        \begin{body1}
          An implementation, in Python, of all data structures and algorithms taught in the ECE 250 (data structures) course. 
          Implements data structures such as hash tables, priority queues, binary trees, graphs etc.
          Implements algorithms such as Djikstra's algorithm, Bellman-Ford etc.
        \end{body1}
      }
      \end{section}

      \vspace*{0.5cm}

      \begin{section}
      {
        \begin{heading3}
          \textcolor{darkcyan}{$\bullet$}
        \end{heading3}
      }
      \end{section}

      \vspace*{0.5cm}

      \begin{heading2}
        Relevant Courses
      \end{heading2}

      \begin{section}
      [
        \begin{body2}
            \textcolor{darkgray}{\hspace{6pt}CS 488 \newline
            ECE 415 \newline
            ECE 423 \newline
            ECE 429 \newline
            ECE 457B \newline
            ECE 327 \newline
            ECE 254}
        \end{body2}
      ]
      {
        \begin{body2}
            \textcolor{darkgray}{Computer Graphics \newline
            Multimedia Communications \newline
            Embedded Computer Systems \newline
            Computer Architecture \newline
            Fundamentals of Computational Intelligence \newline
            Digital Hardware Systems \newline
            Operating Systems \& Systems Programming}
        \end{body2}
      }
      \end{section}
      
      \vspace*{0.5cm}

      \begin{section}
      {
        \begin{heading3}
          \textcolor{darkcyan}{$\bullet$}
        \end{heading3}
      }
      \end{section}

      \vspace*{0.5cm}

      \begin{heading2}
        Areas of Interest
      \end{heading2}

      \begin{minipage}[t]{\columnwidth}
        \begin{body2}
          \textcolor{darkgray}{Operating systems \newline
            CPU architecture \newline
            Embedded systems \newline
            3D graphics programming \newline
            C/C++/ARM assembly programming}
        \end{body2}
      \end{minipage}

    \end{minipage}

  \end{multicols}

\end{document}
